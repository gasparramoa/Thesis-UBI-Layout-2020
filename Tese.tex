\documentclass[11pt,twoside]{estiloUBI}


% Adaptação de Gaspar Ramôa ;)

\include{formatacaoUBI}

\usepackage[export]{adjustbox}
\usepackage{amssymb,siunitx}
\usepackage{fontspec}
\usepackage{arydshln}
\setmainfont[Ligatures=TeX, 
 BoldFont={georgiab.ttf}, 
 ItalicFont={georgiai.ttf},
 BoldItalicFont={georgiaz.ttf}]{georgia.ttf}

\usepackage{acronym}
%%Comentar a linha seguinte se escrever a tese em inglês
%\portugues


%%Para índice remissivo
\makeindex


%%Escolher tipo de letra a usar:
%\usepackage{lmodern}												%Latin modern
%\usepackage{palatino}												%Palatino
%\usepackage{times}												    %Times


%%O comando seguinte insere o nome da tese no cabeçalho das páginas (comentar se não for pretendido)
%\cabecalho{Artificial Vision for Humans} Gaspar - Já não se usa



\begin{document}

%%O comando seguinte insere o espaçamento de 1.5 linhas
\onehalfspacing

%%Página de rosto
\pagenumbering{roman}
\begin{titlepage}
\begin{center}

\rostosubtit \textbf{}\\


\begin{flushright}
 \includegraphics[height=3.0cm]{ubi}\\
%\rostoubi UNIVERSIDADE DA BEIRA INTERIOR\\
%\rostofac Engenharia\\
\end{flushright}

\begin{flushright}
\vspace{6.0cm}

\rostotitulo \textbf{Titulo da Tese} \\
\rostosubtit \textbf{}\\

\vspace{1.8cm}

\rostonomes \textbf{Nome completo}\\

\vspace{1.4cm}

\rostooutros Dissertação para obtenção do Grau de Mestre em\\
\rostonomes \textbf{Engenharia Informática}\\
\rostooutros (2º ciclo de estudos)\\

\vspace{3.3cm}

\rostooutros Orientador: Prof. (...)\\
Co-orientador: Prof. Doutor (...)\\

\vspace{1.4cm}

\rostooutros \textbf{junho de 2020}

\end{flushright}

\end{center}
\end{titlepage}



%\dominitoc


%%Numeração das páginas
\pagestyle{fancy}


%%O comando a seguir gera uma página após a de rosto com cabeçalho e rodapé
\cleardoublepage

%%O comando a seguir permite que as costas da página de rosto não inclua cabeçalho mas rodapé (escolher entre este e outro)
%\newpage\mbox{}\thispagestyle{plain}\fancyhead{}


%%Dedicatória
\newpage 
\section*{\titulos{Dedicatória}}
\vspace{0.5cm}
Inserir dedicatória (opcional)
\cleardoublepage
%\newpage 	
%\mbox{}
%\vfil
%\begin{center}
%Dedicated to...
%\end{center}
%\vfil
%\eject
%\cleardoublepage


%%Agradecimentos 
\newpage 	
\section*{\titulos{Agradecimentos}}
\vspace{0.5cm}
Agradecer a quem de direito (opcional)
\cleardoublepage


%%Prefácio 
\newpage 	
\section*{\titulos{Prefácio}}
\vspace{0.5cm}
Opcional
\cleardoublepage


%%Resumo+palavras-chave
\newpage 	
\section*{\titulos{Resumo}}
\vspace{0.5cm}
Resumo do trabalho
 
\vspace{2.2cm}
{\titulos{Palavras-chave}}
% As que tu quiseres...
\vspace{0.8cm}

\cleardoublepage


%%Resumo alargado 
\newpage 	
\section*{\titulos{Resumo alargado}}
\vspace{0.5cm}
Unicamente para teses em língua estrangeira
\cleardoublepage


%%abstract+keywords
\newpage 	
\section*{\titulos{Abstract}}
\vspace{0.5cm}
Abstract in English

\vspace{2.2cm}
{\titulos{Keywords}}
 
\vspace{0.8cm}
Computer Vision, Visually impaired people...
\cleardoublepage


%%Índice
\tableofcontents





%%Lista de figuras
\listoffigures
\cleardoublepage	


%%Lista de tabelas
\listoftables
\cleardoublepage


%Abreviaturas 
\newpage
\section*{\titulos{Acronyms}}
\vspace{1.3cm}
% TODO: ENSURE ALPHABETICAL ORDER 
\begin{acronym}[GPU] % Maior acrónimo, para ficar tudo tabulado!
\acro{AI}{Artificial Intelligence}
% ... coloquem os vossos
\end{acronym}
 \cleardoublepage
  

%% Os capitulos são inseridos a partir daqui 
 
\mainmatter

\include{Intro}
\include{relativeWork}


%\include{Exemplos}



%% Fim da inserção dos capitulos


%% Inicio Bibliografia
\cleardoublepage
\phantomsection
\addcontentsline{toc}{chapter}{Bibliografia}
%%%%%%%%%%%%%%%%
% Escolher entre as duas opcções
%
% A primeira é a aconselhada pelo despacho reitoral
% A segunda é a utilizada pelo IEEE
%
%Primeira opcção
\bibliographystyle{estilo-biblio}				%Estilo bibliografia com nomes
\bibliography{bibliografia}					%Entrada biblbiografia aconselhada com nomes
%
% Segunda opcção
%\bibliographystyle{IEEEtran}					%Estilo bibliografia IEEE
%\bibliography{IEEEabrv,bibliografia}				%Entrada bibliografia aconselhada para IEEE
%% Fim Bibliografia


%%Anexos
\appendix
 
\include{Anexos}
\cleardoublepage


%%Glossário
\newpage
\section*{\titulos{Glossário}}
\vspace{0.5cm}
	\noindent\begin{tabularx}{\linewidth}{l p{0.5cm} Y}
	\LaTeX & & Conjunto de macros para o processador de textos \TeX, utilizado amplamente para a produção de textos matemáticos e científicos devido à sua alta qualidade tipográfica.\cr
	\end{tabularx}
\cleardoublepage



%%Inserir índice remissivo
\printindex

\end{document}
